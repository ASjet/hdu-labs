\documentclass[a4paper]{ctexart}
\usepackage[utf8]{inputenc}
\usepackage[a4paper]{geometry}
\usepackage{graphicx}
\usepackage{float}
\usepackage{hyperref}
\usepackage[heading = false]{ctex}
\usepackage{xcolor}
\usepackage{fontspec}
\usepackage{listings}
\usepackage{color}

\definecolor{dkgreen}{rgb}{0,0.6,0}
\definecolor{gray}{rgb}{0.5,0.5,0.5}
\definecolor{mauve}{rgb}{0.58,0,0.82}

\lstset{
  frame=tb,
  aboveskip=3mm,
  belowskip=3mm,
  showstringspaces=false,
  columns=flexible,
  basicstyle = \ttfamily,
  numbers=none,
  numberstyle=\tiny\color{gray},
  keywordstyle=\color{blue},
  commentstyle=\color{dkgreen},
  stringstyle=\color{mauve},
  breaklines=true,
  breakatwhitespace=true,
  tabsize=3
}
\pagestyle{plain}
\geometry{top=1.0cm, bottom=2.0cm}

\begin{document}
  \begin{titlepage}
      \songti
      \begin{center}
        \vspace*{2cm}
        \includegraphics[width=0.7\textwidth]{../HDU.png}\\
        \vspace*{1cm}
        {\fontsize{36pt}{0}
          \textbf{机器学习实验\\报\quad 告\\}
        }
        \vspace*{12cm}
        {\fontsize{18pt}{0}
          \makebox[80pt]{\textbf{实验名称}} \underline{\makebox[250pt]{\Large 基于聚类的整图分割}}\\
          \vspace*{0.5cm}
          \makebox[80pt]{\textbf{学\qquad 院}} \underline{\makebox[250pt]{\Large 通信工程学院}}\\
          \vspace*{0.5cm}
          \makebox[80pt]{\textbf{专\qquad 业}} \underline{\makebox[250pt]{\Large xxxx}}\\
          \vspace*{0.5cm}
          \makebox[80pt]{\textbf{学\qquad 号}} \underline{\makebox[250pt]{\Large xxxx}}\\
          \vspace*{0.5cm}
          \makebox[80pt]{\textbf{学生姓名}} \underline{\makebox[250pt]{\Large xxx}}\\
        }
      \end{center}
  \end{titlepage}

  \CTEXsetup[format={\Large\bfseries}]{section}

  \newpage
  \section{实验目的}
    \begin{enumerate}
      \item 理解无监督学中Kmeans算法原理
      \item 掌握Sklearn实现基于Kmeans的图像分割
    \end{enumerate}

  \section{实验内容与要求}
    \begin{enumerate}
      \item 利用K-means聚类算法对图像像素点颜色进行聚类实现简单的图像分割
      \item 输出:同一聚类中的点使用相同的颜色标记,不同聚类颜色不同
      \item 技术路线:\verb|sklearn.cluster.KMeans|
    \end{enumerate}

  \section{实验程序与结果}
  \subsection{程序代码}
  \lstinputlisting[language=Python]{lab8.py}
  \subsection{运行结果}
  由于KMeans只能输出聚类结果的标签值,因此要可视化结果需要将聚类标签映射到颜色空间上。
  因为标签是一维的,所以选择了HSI的H通道(色调)作为映射的颜色空间,S和I的值都设置为1.0
  \begin{figure}[H]
    \includegraphics[width=1.0\textwidth]{fig/band.png}
    \caption{HSI色带}
  \end{figure}
  \newpage
  \begin{figure}[H]
    \includegraphics[width=1.0\textwidth]{fig/res.png}
    \caption{标签对应的颜色}
    \includegraphics[width=1.0\textwidth]{fig/res2.png}
    \caption{可视化聚类结果}
  \end{figure}

  \section{实验问题解答与体会}
  本次实验使用了聚类算法对灰度图像进行分割,加深了我对Kmeans应用场景的理解

\end{document}
