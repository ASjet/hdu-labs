\documentclass[a4paper]{ctexart}
\usepackage[utf8]{inputenc}
\usepackage[a4paper]{geometry}
\usepackage{graphicx}
\usepackage{float}
\usepackage{hyperref}
\usepackage[heading = false]{ctex}
\usepackage{xcolor}
\usepackage{fontspec}
\usepackage{listings}
\pagestyle{plain}
\geometry{top=1.0cm, bottom=2.0cm}
\setmonofont{Cascadia Code PL}
\lstset{
    basicstyle = \ttfamily,
    commentstyle = \itshape,
    numbers = left,
    numberstyle = \zihao{-5}\ttfamily,
    frame = lrtb
}

\begin{document}
  \begin{titlepage}
      \songti
      \begin{center}
        \vspace*{2cm}
        {\fontsize{24pt}{0}
          \textbf{计算机通信与网络\\实验报告\\}
        }
        \vspace*{10cm}
        {\fontsize{16pt}{0}
          \textbf{\ 姓\quad 名\ }: \underline{\makebox[100pt]{xxx}}\\
          \textbf{\ 学\quad 号\ }: \underline{\makebox[100pt]{\Large xxxx}}\\
          \textbf{\ 班\quad 级\ }: \underline{\makebox[100pt]{\Large xxxx}}\\
          \textbf{上课时间}: \underline{\makebox[100pt]{\Large yyyy.mm.dd}}\\
        }
        \vspace*{7cm}
        {\fontsize{16pt}{0}
          \textbf{实验名称: {\Large XXX}}
        }
      \end{center}
  \end{titlepage}

  \CTEXsetup[format={\Large\bfseries}]{section}

  \newpage
  \section{实验目的}
    \begin{itemize}
      \item
    \end{itemize}

  \section{实验内容与要求}
    \begin{enumerate}
      \item 1
    \end{enumerate}

    \paragraph{实验原理}
    \begin{enumerate}
      \item 1
    \end{enumerate}

    \newpage
    \section{实验程序与结果}
    \subsection{运行结果}
    % \begin{figure}[H]
    % \includegraphics[width=0.7\textwidth]{fig/XXX.png}
    % \caption{XXX}
    % \end{figure}

    \subsection{程序代码}
    见\hyperlink{appendix}{附录}

    \section{实验结果分析}
    XXX

    \newpage
    \section{实验问题解答与体会}
    XXX

    \newpage
    \appendix
    \hypertarget{appendix}{}
    \section*{附录}

    \subsection*{Name}
    % \lstinputlisting[language=XXX, caption=XXX.*]{code/XXX}
\end{document}
