\documentclass[a4paper]{ctexart}
\usepackage[utf8]{inputenc}
\usepackage[a4paper]{geometry}
\usepackage{graphicx}
\usepackage{float}
\usepackage{hyperref}
\usepackage[heading = false]{ctex}
\usepackage{xcolor}
\usepackage{fontspec}
\usepackage{listings}
\pagestyle{plain}
\geometry{top=1.0cm, bottom=2.0cm}
\setmonofont{Cascadia Code PL}
\lstset{
    basicstyle = \ttfamily,
    commentstyle = \itshape,
    numbers = left,
    numberstyle = \zihao{-5}\ttfamily,
    frame = lrtb
}

\begin{document}
  \begin{titlepage}
      \songti
      \begin{center}
        \vspace*{2cm}
        {\fontsize{24pt}{0}
          \textbf{计算机通信与网络\\实验报告\\}
        }
        \vspace*{10cm}
        {\fontsize{16pt}{0}
          \textbf{\ 姓\quad 名\ }: \underline{\makebox[100pt]{xxx}}\\
          \textbf{\ 学\quad 号\ }: \underline{\makebox[100pt]{\Large xxxx}}\\
          \textbf{\ 班\quad 级\ }: \underline{\makebox[100pt]{\Large xxxx}}\\
          \textbf{上课时间}: \underline{\makebox[100pt]{\Large yyyy.mm.dd}}\\
        }
        \vspace*{7cm}
        {\fontsize{16pt}{0}
          \textbf{实验名称: {\Large Basic Network Configuration}}
        }
      \end{center}
  \end{titlepage}

  \CTEXsetup[format={\Large\bfseries}]{section}

  \newpage
  \section{实验目的}
    \begin{itemize}
      \item Use commandline to check network configuration.
    \end{itemize}

  \section{实验内容与要求}
    \begin{enumerate}
      \item What is the purpose of \textbf{NET VIEW}:\\
      显示局域网内所有的主机名(hostname)

      \item List the machines after you have typed \textbf{NET VIEW} from the command prompt
      \begin{figure}[H]
        \includegraphics*[width=1.0\textwidth]{fig/NETVIEW.png}
        \caption{NET VIEW输出}
      \end{figure}

      \item Record the following TCP/IP information for this computer
      \begin{figure}[H]
        \includegraphics*[width=1.0\textwidth]{fig/ip.png}
        \caption[]{TCP/IP configuration in Linux}
      \end{figure}
      \makebox[100pt]{\textbf{IP Address}} \underline{\makebox[80pt]{192.168.1.103}}\\
      \makebox[100pt]{\textbf{Subnet Mask}} \underline{\makebox[80pt]{255.255.255.0}}\\
      \makebox[100pt]{\textbf{Default Gateway}} \underline{\makebox[80pt]{192.168.1.1}}

      \newpage
      \item Compare the TCP/IP configuration of this computer to others on the LAN
      \begin{figure}[H]
        \includegraphics*[width=1.0\textwidth]{fig/ip2.png}
        \caption[]{TCP/IP configuration in PC and RaspberryPi4 both in Linux}
      \end{figure}
      \makebox[70pt]{\textbf{IP Address 1}} \underline{\makebox[80pt]{192.168.1.103}}\\
      \makebox[70pt]{\textbf{IP Address 2}} \underline{\makebox[80pt]{192.168.1.107}}\\
      \textbf{Are there any similarities?}\\
      \underline{These two addresses are in the same subnet 192.168.1.0/24}\\
      \textbf{What is similar about the default gateway?}\\
      \underline{They have the same gateway router at 192.168.1.1}

      \newpage
      \item Mark the Physical Address(MAC) and the NIC model(Description)
      \begin{figure}[H]
        \includegraphics*[width=1.0\textwidth]{fig/mac.png}
        \caption[]{MAC Address in PC and RaspberryPi4}
        \includegraphics*[width=1.0\textwidth]{fig/nic.png}
        \caption[]{NIC description in PC}
        \includegraphics*[width=1.0\textwidth]{fig/macman.png}
        \caption[]{NIC manufacturer(OEM)}
      \end{figure}
      \makebox[90pt]{\textbf{MAC Address}} \underline{\makebox[80pt]{c8:f7:50:7d:a6:fe}}\\
      \makebox[90pt]{\textbf{NIC Description}} \underline{Qualcomm Atheros Killer E2500 Gigabit Ethernet Controller``}

      \item While not a requirement, most LAN administrators try to standardize components like NICs.
      Therefore, it would not be surprising to find all machines share the first three Hex pairs in the adapter address.
      These three pairs identity the manufacturer of the adapter.\\
      \textbf{c8:f7:50} Dell Inc.\\
      \textbf{14:7d:da} Apple Inc\\
      \textbf{e4:5f:01} Raspberry Pi Trading Ltd

      \newpage
      \item Write down the IP addresses of any servers listed(if any)
      \begin{figure}[H]
        \includegraphics*[width=1.0\textwidth]{fig/dhcp.png}
        \caption[]{DHCP and DNS Server}
      \end{figure}
      \makebox[70pt]{\textbf{DHCP Server}} \underline{\makebox[80pt]{192.168.1.1}}\\
      \makebox[70pt]{\textbf{DNS Server}} \underline{\makebox[80pt]{208.67.222.222}}\\
      \makebox[70pt]{\textbf{DNS Server}} \underline{\makebox[80pt]{208.67.220.220}}

      \item Write down the computer Hostname
      \begin{figure}[H]
        \centering \includegraphics*[width=0.5\textwidth]{fig/hostname.png}
        \caption[]{Hostname of PC and RaspberryPi4}
      \end{figure}
      \underline{Anduril(192.168.1.103)}

      \item Write down the Hostname of a couple other computers\\
      \underline{Artemis.local(192.168.1.100)、RaspberryPi4(192.168.1.107)}\\
      Do all of the servers and workstations share the same network portion of
      the IP address as the student workstation? \underline{Yes except DNS Server}

      \newpage
      \item The following figure shows the successful results of ping to this IP address.
      \begin{figure}[H]
        \includegraphics*[width=1.0\textwidth]{fig/ping1.png}
        \caption[]{From Anduril ping Artemis.local}
      \end{figure}
      Is the \textbf{ping} successful? \underline{Yes}\\
      If a second networked computer is avaliable, try to \textbf{ping} the IP address of the
      second machine. Note the results:
      \begin{figure}[H]
        \includegraphics*[width=1.0\textwidth]{fig/ping2.png}
        \caption[]{From Anduril ping RaspberryPi4}
      \end{figure}

      \newpage
      \item Try to \textbf{ping} the IP address of any DHCP and/or DNS servers listed
      in the last exercise. If this works for either server, and they are not in the network,
      what does this indicate?
      \begin{figure}[H]
        \includegraphics*[width=1.0\textwidth]{fig/pinggateway.png}
        \caption[]{ping DHCP Server}
        \includegraphics*[width=1.0\textwidth]{fig/pingdns.png}
        \caption[]{ping DNS Server}
      \end{figure}
      \underline{The DHCP/DNS servers are not in the same network while the ping succeed means that}\\
      \underline{there is one or more routers connecting the local network and the networks where the}\\
      \underline{DHCP/DNS servers located, and the DHCP/DNS packets are forwarded by the routes}\\
      Was the \textbf{ping} successful? \underline{Yes}

      \newpage
      \item What is the IP address of your default gateway?\
      \underline{192.168.1.1}\\
      \begin{figure}[H]
        \includegraphics*[width=1.0\textwidth]{fig/pinggateway.png}
        \caption[]{ping gateway Server}
      \end{figure}
      Try to \textbf{ping} the IP address of the default gateway. Was it successful?
      \underline{Yes}

      \item Tracert a website and explain what you have found.
      \begin{figure}[H]
        \includegraphics*[width=1.0\textwidth]{fig/traceroute.png}
        \caption{traceroute login.hdu.edu.cn}
      \end{figure}
      从图中可以看出,本机和login.hdu.edu.cn服务器在两个不同的子网

      本机所处的局域网是10.*.*.*,login.hdu.edu.cn服务器所在的局域网是192.168.*.*,
      这两个局域网之间通过一个路由器(210.32.39.251)相连接

    \end{enumerate}

    \newpage
\end{document}
