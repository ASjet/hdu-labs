\documentclass[a4paper]{ctexart}
\usepackage[utf8]{inputenc}
\usepackage[a4paper]{geometry}
\usepackage{graphicx}
\usepackage{float}
\usepackage{hyperref}
\usepackage[heading = false]{ctex}
\usepackage{xcolor}
\usepackage{fontspec}
\usepackage{listings}
\pagestyle{plain}
\geometry{top=1.0cm, bottom=2.0cm}
\lstset{
    basicstyle = \ttfamily,
    commentstyle = \itshape,
    numbers = left,
    numberstyle = \zihao{-5}\ttfamily,
    frame = lrtb
}

\begin{document}
  \begin{titlepage}
      \songti
      \begin{center}
        \vspace*{1cm}
        \includegraphics[width=0.9\textwidth]{../HDU.png}\\
        \vspace*{2.5cm}
        {\fontsize{24pt}{0}
          《图像处理实验》\\
          \fontsize{36pt}{0}
          \vspace*{1cm}
          实验报告\\
        }
        \vspace*{3cm}
        {\fontsize{18pt}{0}
          \makebox[80pt]{实验名称:} \makebox[100pt]{\Large 图像平滑}\\
          \vspace*{2cm}
          \makebox[80pt]{姓\qquad \quad 名:} \makebox[100pt]{\Large xxx}\\
          \vspace*{0.5cm}
          \makebox[80pt]{学\qquad \quad 号:} \makebox[100pt]{\Large xxxx}\\
          \vspace*{0.5cm}
          \makebox[80pt]{专\qquad \quad 业:} \makebox[100pt]{\Large xxxx}\\
          \vspace*{0.5cm}
          \makebox[80pt]{实验时间:} \makebox[100pt]{\Large yyyy.mm.dd}\\
          \vspace*{5cm}
          杭州电子科技大学\\通信工程学院
        }
      \end{center}
  \end{titlepage}

  \CTEXsetup[format={\Large\bfseries}]{section}

  \newpage
  \section{实验目的}
  \begin{enumerate}
    \item 掌握图像滤波算法的基本原理
    \item 了解不同滤波算法适用场景
  \end{enumerate}

  \section{实验内容}
  \begin{enumerate}
    \item 读取图像I, 使用MATLAB工具箱中的函数为图像I增加椒盐噪声,获得新图像I1
    \item 为图像I增加高斯噪声,获得新图像I2
    \item 使用模板大小为3*3、7*7、15*15的中值滤波器分别处理图像I1,I2, 获得处理后的图像I3-I8
    \item 使用模板大小为3*3、7*7、15*15的均值滤波器分别处理图像I1,I2, 获得处理后的图像I9-I14
    \item 显示并对比各原图与滤波后的图像
  \end{enumerate}

  \section{算法设计}
  \begin{enumerate}
    \item \textbf{什么是图像平滑?}

    图像平滑是值去除图像中噪声或噪点的操作,因为这种操作通过去掉噪点降低了图像的变化率,
    因此叫做图像平滑
    \item \textbf{什么是椒盐噪声?什么是高斯噪声?}
    \paragraph*{椒盐噪声}椒盐噪声也称为脉冲噪声,是图像中经常见到的一种噪声,它是一种随机出现的白点或者黑点,
    可能是亮的区域有黑色像素或是在暗的区域有白色像素(或是两者皆有)。
    \paragraph*{高斯噪声}高斯噪声(Gaussian noise)是一种具有正态分布(也称作高斯分布)概率密度函数的噪声。
    换句话说,高斯噪声的值遵循高斯分布或者它在各个频率分量上的能量符合高斯分布。
    \item \textbf{描述不同图像平滑方法对不同噪声的处理效果,并分析原因}
    \paragraph*{中值滤波}对椒盐噪声处理效果较好,对高斯噪声处理效果较差。

    中值滤波是一种取滤波核中所有元素的中位数的滤波操作。
    这种滤波操作基于一种假设,即图像中某一像素的灰度值与其邻域的差别不是很大,
    当噪声的灰度值是离群点时,通过选择邻域内所有灰度值的中位数作为滤波结果,即可过滤掉这种模式的噪声。\\
    由椒盐噪声的特性可知,中值滤波对于椒盐噪声的处理效果较理想。

    由于中位数仍带有高斯噪声分量,因此中值滤波对于高斯噪声的处理效果较差。
    \paragraph*{均值滤波}对椒盐噪声处理效果较差,对高斯噪声处理效果较好。

    均值滤波是一种取滤波核中所有元素的平均数的滤波操作。
    由于高斯分布具有均值为零的特性,因此可以认为像素与其邻域的灰度值之和在叠加高斯噪声前后基本保持不变,
    即可认为均值滤波对叠加高斯噪声前后的图像处理结果基本一致,
    因此只要控制好滤波核的大小就能达到对高斯噪声较好的处理效果。

    对于椒盐噪声来说,均值滤波使得图像中的噪声被平均分配到了周围的像素当中,
    因此当滤波核较小时反而会扩大椒盐噪声对图像的影响;尽管当滤波核较大时能分摊掉离群的噪点,
    但较大的滤波核本身就会导致图像变得模糊,因此均值滤波对椒盐噪声的处理效果较差。
    \item \textbf{在模板作用于图像中时,图像边缘区域有几种处理方式,分别是什么?
    在本实验中使用了哪种图像边缘处理方式?}
    \paragraph*{Constant} iiiiii|abcdefgh|iiiiiii(i为指定的常量)
    \paragraph*{Replicate} aaaaaa|abcdefgh|hhhhhhh
    \paragraph*{Reflect} fedcba|abcdefgh|hgfedcb
    \paragraph*{Warp} cdefgh|abcdefgh|abcdefg
    \paragraph*{Transparent} uvwxyz|abcdefgh|ijklmno

    在本实验中使用了填充常量的方法(Constant),填充值为0
  \end{enumerate}

  \newpage
  \section{运行结果与分析}
  \subsection*{椒盐噪声}
  \begin{figure}[H]
    \includegraphics*[width=1.0\textwidth]{fig/salt_noise.png}
    \caption{原图(左); 椒盐噪声(右)}
    \includegraphics*[width=1.0\textwidth]{fig/salt_median.png}
    \caption{3*3中值滤波(左); 7*7中值滤波(中); 15*15中值滤波(右)}
    \includegraphics*[width=1.0\textwidth]{fig/salt_blur.png}
    \caption{3*3均值滤波(左); 7*7均值滤波(中); 15*15均值滤波(右)}
  \end{figure}

  \newpage
  \subsection*{高斯噪声}
  \begin{figure}[H]
    \includegraphics*[width=1.0\textwidth]{fig/gauss_noise.png}
    \caption{原图(左); 高斯噪声(右)}
    \includegraphics*[width=1.0\textwidth]{fig/gauss_median.png}
    \caption{3*3中值滤波(左); 7*7中值滤波(中); 15*15中值滤波(右)}
    \includegraphics*[width=1.0\textwidth]{fig/gauss_blur.png}
    \caption{3*3均值滤波(左); 7*7均值滤波(中); 15*15均值滤波(右)}
  \end{figure}

  \section{实验小结}
  本次实验比较了中值滤波和均值滤波对不同种类的噪声的滤波结果,
  让我直观地看到了不同滤波方法对图像的影响以及其适合处理的噪声,
  加深了对图像处理滤波方法的理解。

\end{document}
