\documentclass[a4paper]{ctexart}
\usepackage[utf8]{inputenc}
\usepackage[a4paper]{geometry}
\usepackage{graphicx}
\usepackage{float}
\usepackage{hyperref}
\usepackage[heading = false]{ctex}
\usepackage{xcolor}
\usepackage{fontspec}
\usepackage{listings}
\pagestyle{plain}
\geometry{top=2.0cm, bottom=2.0cm}
\lstset{
    basicstyle = \ttfamily,
    commentstyle = \itshape,
    numbers = left,
    numberstyle = \zihao{-5}\ttfamily,
    frame = lrtb
}

\begin{document}
  \begin{titlepage}
      \songti
      \begin{center}
        \includegraphics[width=0.9\textwidth]{../HDU.png}\\
        \vspace*{2.5cm}
        {\fontsize{24pt}{0}
          《图像处理实验》\\
          \fontsize{36pt}{0}
          \vspace*{1cm}
          实验报告\\
        }
        \vspace*{3cm}
        {\fontsize{18pt}{0}
          \makebox[80pt]{实验名称:} \makebox[100pt]{\Large 彩色空间变换}\\
          \vspace*{2cm}
          \makebox[80pt]{姓\qquad \quad 名:} \makebox[100pt]{\Large xxx}\\
          \vspace*{0.5cm}
          \makebox[80pt]{学\qquad \quad 号:} \makebox[100pt]{\Large xxxx}\\
          \vspace*{0.5cm}
          \makebox[80pt]{专\qquad \quad 业:} \makebox[100pt]{\Large xxxx}\\
          \vspace*{0.5cm}
          \makebox[80pt]{实验时间:} \makebox[100pt]{\Large yyyy.mm.dd}\\
          \vspace*{5cm}
          杭州电子科技大学\\通信工程学院
        }
      \end{center}
  \end{titlepage}

  \CTEXsetup[format={\Large\bfseries}]{section}

  \newpage
  \section{实验目的}
  掌握使用Matlab进行彩色图像处理的方法,加深对彩色空间和彩色图像的理解

  \section{实验内容}
  \begin{enumerate}
    \item 读取任一图像,同屏显示彩色图像原图及各通道图像
    \item 编写程序实现RGB-HSI空间的相互转换,同屏显示原图与HSI空间各通道图像
    \item 编写程序在不改变色调的情况下,将原图亮度提高20\%,并同屏显示原图与变换后的图像
    \item 编写程序在不改变色调的情况下,将原图饱和度提高20\%,并同屏显示原图与变换后的图像
  \end{enumerate}

  \section{算法设计}
  \begin{enumerate}
    \item \textbf{说明RGB图像各通道的含义及取值范围}\\
    RGB图像包含R、G、B三个通道,分别表示红色(Red)、绿色(Green)和蓝色(Blue)的值,
    值越大对应的颜色就越亮\\
    对任一通道来说,当通道数据类型为整型时取值范围为[0,255],当类型为浮点型时取值范围为[0,1]
    \item \textbf{说明HSI图像各通道的含义以及取值范围}\\
    HSI图像包含H、S、I三个通道,分别表示色调(Hue), 饱和度(Saturation)和亮度(Lightness)\\
    其中色调H决定颜色,不同的值表示不同的颜色;取值范围为[0,360](度)或[0,$2\pi$](弧度)\\
    饱和度S决定了颜色的纯度,值越大颜色越纯,值越小颜色越接近灰色;取值范围为[0,1](浮点型)\\
    亮度L决定了颜色的明亮程度,值越大颜色越亮,值越小颜色越暗;取值范围为[0,1](浮点型)
  \end{enumerate}

  \newpage
  \section{运行结果与分析}
  \subsection*{实验内容1}
  \begin{figure}[H]
    \centering
      \includegraphics*[width=0.7\textwidth]{fig/R-G-B.png}
      \caption{从上到下从左到右依次为原图、R通道、G通道和B通道}
  \end{figure}
  \subsection*{实验内容2}
  \begin{figure}[H]
    \centering
      \includegraphics*[width=0.7\textwidth]{fig/H-S-I.png}
      \caption{从上到下从左到右依次为原图、H通道、S通道和I通道}
  \end{figure}
  \subsection*{实验内容3}
  \begin{figure}[H]
    \centering
      \includegraphics*[width=1\textwidth]{fig/bright20.png}
      \caption{左图为原图,右图为原图提高20\%亮度的结果}
  \end{figure}
  \subsection*{实验内容4}
  \begin{figure}[H]
    \centering
      \includegraphics*[width=1\textwidth]{fig/sat20.png}
      \caption{左图为原图,右图为原图提高20\%饱和度的结果}
  \end{figure}

  \section{实验小结}
  在对图像进行彩色空间转换时,应用转换公式之前要注意对像素值进行处理,如进行归一化等,
  避免在不同取值范围的像素之间进行运算

  当使用HSI空间提高图像的亮度和对比度时,不能再将像素值重新映射到[0,1]范围内,
  而应将大于1的值置为1,否则会导致结果无变化或不明显

\end{document}
