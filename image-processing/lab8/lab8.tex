\documentclass[a4paper]{ctexart}
\usepackage[utf8]{inputenc}
\usepackage[a4paper]{geometry}
\usepackage{graphicx}
\usepackage{float}
\usepackage{hyperref}
\usepackage[heading = false]{ctex}
\usepackage{xcolor}
\usepackage{fontspec}
\usepackage{listings}
\pagestyle{plain}
\geometry{top=1.0cm, bottom=2.0cm}
\lstset{
    basicstyle = \ttfamily,
    commentstyle = \itshape,
    numbers = left,
    numberstyle = \zihao{-5}\ttfamily,
    frame = lrtb
}

\begin{document}
  \begin{titlepage}
      \songti
      \begin{center}
        \vspace*{1cm}
        \includegraphics[width=0.9\textwidth]{../HDU.png}\\
        \vspace*{2.5cm}
        {\fontsize{24pt}{0}
          《图像处理实验》\\
          \fontsize{36pt}{0}
          \vspace*{1cm}
          实验报告\\
        }
        \vspace*{3cm}
        {\fontsize{18pt}{0}
          \makebox[80pt]{实验名称:} \makebox[150pt]{\Large 应用案例2-道路检测}\\
          \vspace*{2cm}
          \makebox[80pt]{姓\qquad \quad 名:} \makebox[100pt]{\Large xxx}\\
          \vspace*{0.5cm}
          \makebox[80pt]{学\qquad \quad 号:} \makebox[100pt]{\Large xxxx}\\
          \vspace*{0.5cm}
          \makebox[80pt]{专\qquad \quad 业:} \makebox[100pt]{\Large xxxx}\\
          \vspace*{0.5cm}
          \makebox[80pt]{实验时间:} \makebox[100pt]{\Large 2022.XX.XX}\\
          \vspace*{5cm}
          杭州电子科技大学\\通信工程学院
        }
      \end{center}
  \end{titlepage}

  \CTEXsetup[format={\Large\bfseries}]{section}

  \newpage
  \section{实验目的}
    掌握图像处理算法在道路检测或车道线检测中的应用

  \section{实验内容}
    根据前几次课内容,设计算法,实现图像中道路的检测或车道线的检测。
    要求标记出道路或车道线区域(道路或车道线标记1,其他标记0)

  \section{算法设计}
    \begin{enumerate}
      \item 给出算法的设计思路,说明每部分算法的作用
      \begin{enumerate}
        \item 读取图像并灰度化处理,使用Canny提取边缘,获得车道线和道路的轮廓信息
        \item 对轮廓做霍夫变换,得到若干直线
        \item 根据透视原理用角度过滤掉非行进方向的直线
        \item 根据直线角度信息(小于或大于90度)区分左侧道路线和右侧道路线
        \item 以剩余直线为界限,利用搜索算法填充各个区域,得到不同区域的Mask
      \end{enumerate}
      \item 分析不同分割方法在道路或车道线检测问题中有哪些优缺点
    \end{enumerate}

  \section{运行结果与分析}
    \begin{figure}[H]
      \includegraphics*[width=1.0\textwidth]{fig/raw.png}
      \caption{原始图像}
    \end{figure}
    \newpage
    \begin{figure}[H]
      \includegraphics*[width=1.0\textwidth]{fig/gray.png}
      \caption{灰度图}
      \includegraphics*[width=1.0\textwidth]{fig/edge.png}
      \caption{Canny边缘}
      \includegraphics*[width=1.0\textwidth]{fig/block.png}
      \caption{霍夫变换检测路面和车道线边缘}
    \end{figure}
    \newpage
    \begin{figure}[H]
      \includegraphics*[width=1.0\textwidth]{fig/road.png}
      \caption{路面区域(白色区块)}
      \includegraphics*[width=1.0\textwidth]{fig/line.png}
      \caption{车道线区域(黑色区块)}
    \end{figure}

  \section{实验小结}
  这次实验使用了理论课上没有讲的霍夫变换,通过搜索参数空间能够将离散的像素点与连续的曲线方程联系起来,
  让我体会到了数学之美。

\end{document}
