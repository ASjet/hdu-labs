\documentclass[a4paper]{ctexart}
\usepackage[utf8]{inputenc}
\usepackage[a4paper]{geometry}
\usepackage{graphicx}
\usepackage{float}
\usepackage{hyperref}
\usepackage[heading = false]{ctex}
\usepackage{xcolor}
\usepackage{fontspec}
\usepackage{listings}
\usepackage{multicol}

\pagestyle{plain}
\geometry{top=1.0cm, bottom=2.0cm}
\lstset{
    basicstyle = \ttfamily,
    commentstyle = \itshape,
    numbers = left,
    numberstyle = \zihao{-5}\ttfamily,
    frame = lrtb
}

\begin{document}
  \begin{titlepage}
      \songti
      \begin{center}
        \vspace*{1cm}
        \includegraphics[width=0.9\textwidth]{../HDU.png}\\
        \vspace*{2.5cm}
        {\fontsize{24pt}{0}
          《图像处理实验》\\
          \fontsize{36pt}{0}
          \vspace*{1cm}
          实验报告\\
        }
        \vspace*{3cm}
        {\fontsize{18pt}{0}
          \makebox[80pt]{实验名称:} \makebox[100pt]{\Large 图像边缘检测}\\
          \vspace*{2cm}
          \makebox[80pt]{姓\qquad \quad 名:} \makebox[100pt]{\Large xxx}\\
          \vspace*{0.5cm}
          \makebox[80pt]{学\qquad \quad 号:} \makebox[100pt]{\Large xxxx}\\
          \vspace*{0.5cm}
          \makebox[80pt]{专\qquad \quad 业:} \makebox[100pt]{\Large xxxx}\\
          \vspace*{0.5cm}
          \makebox[80pt]{实验时间:} \makebox[100pt]{\Large yyyy.mm.dd}\\
          \vspace*{5cm}
          杭州电子科技大学\\通信工程学院
        }
      \end{center}
  \end{titlepage}

  \CTEXsetup[format={\Large\bfseries}]{section}

  \newpage
  \section{实验目的}
  \begin{enumerate}
    \item 掌握使用微分算子进行图像边缘检测的基本原理
    \item 了解梯度算子、拉普拉斯算子的优缺点
  \end{enumerate}

  \section{实验内容}
  \begin{enumerate}
    \item 使用MATLAB工具箱中的roberts,prewitt,sobel算子以及拉普拉斯算子检测给定lena图像边缘,
    对比实验结果
    \item 根据算法原理编码实现sobel微分算子(非工具箱函数),实现图像的边缘检测
  \end{enumerate}

  \section{代码实现}
  \lstinputlisting[language=Python]{lab5.py}

  \newpage
  \newgeometry{top=1.0cm, bottom=2.0cm, left=1.0cm, right=1.0cm}
  \section{运行结果与分析}
  \begin{figure}[H]
    \includegraphics*[width=1.0\textwidth]{fig/raw.png}
    \caption{原始图片及简单处理}
    \includegraphics*[width=1.0\textwidth]{fig/roberts.png}
    \caption{Roberts边缘检测(阈值=50)}
    \includegraphics*[width=1.0\textwidth]{fig/prewitt.png}
    \caption{Prewitt边缘检测(阈值=50)}
  \end{figure}

  \newpage
  \begin{figure}[H]
    \begin{tabular}{p{0.5\textwidth}p{0.5\textwidth}}
      \centering
      \includegraphics*[width=0.45\textwidth]{fig/laplas.png}
      \caption{Laplacian边缘检测(阈值=50)}
      &
      \centering
      \includegraphics*[width=0.45\textwidth]{fig/canny.png}
      \caption{Canny边缘检测(阈值=50)}
    \end{tabular}

    \includegraphics*[width=1.0\textwidth]{fig/sobel1.png}
    \caption{Opencv库函数实现Sobel边缘检测(阈值=50)}

    \includegraphics*[width=1.0\textwidth]{fig/sobel2.png}
    \caption{根据算法原理实现Sobel边缘检测(阈值=50)}
  \end{figure}

  \newpage
  \newgeometry{top=1.0cm, bottom=2.0cm}
  \section{问题思考}
  \begin{enumerate}
    \item 对比三种梯度算子与拉普拉斯算子边缘检测结果,分析各方法优缺点

    \begin{itemize}
      \item Roberts算子是2*2对角矩阵,检测精度高,但对噪声比较敏感
      \item Prewitt是一种一阶微分算子,包括水平和竖直两个算子。
      算子大小为3*3,能去掉伪边缘,对噪声具有平滑作用,但同时也会丢掉一些细节。
      \item Sobel算子相比Prewitt算子,在不同的位置权重不一样,因此可以降低边缘模糊程度,
      但得到的边缘较粗,而且可能出现伪边缘
      \item 拉普拉斯算子是二阶梯度算子,能够找到变化平滑的边缘,但是对细节不敏感
    \end{itemize}

    \item 给出至少一种处理边缘的方法,使获得的边缘更连续

    对图像做高斯滤波,以减少高斯噪声对边缘检测结果的影响
  \end{enumerate}

  \section{实验小结}
  本次实验比较了多种边缘检测算子的结果,让我对不同边缘检测算子的优缺点有了一个直观的感受。
  同时也亲自实现了Sobel边缘检测算法,让我更加理解了构造算子进行边缘检测的原理。

\end{document}
