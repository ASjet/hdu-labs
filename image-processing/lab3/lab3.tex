\documentclass[a4paper]{ctexart}
\usepackage[utf8]{inputenc}
\usepackage[a4paper]{geometry}
\usepackage{graphicx}
\usepackage{float}
\usepackage{hyperref}
\usepackage[heading = false]{ctex}
\usepackage{xcolor}
\usepackage{fontspec}
\usepackage{listings}
\pagestyle{plain}
\geometry{top=1.0cm, bottom=2.0cm}
\lstset{
    basicstyle = \ttfamily,
    commentstyle = \itshape,
    numbers = left,
    numberstyle = \zihao{-5}\ttfamily,
    frame = lrtb
}

\begin{document}
  \begin{titlepage}
      \songti
      \begin{center}
        \vspace*{1cm}
        \includegraphics[width=0.9\textwidth]{../HDU.png}\\
        \vspace*{2.5cm}
        {\fontsize{24pt}{0}
          《图像处理实验》\\
          \fontsize{36pt}{0}
          \vspace*{1cm}
          实验报告\\
        }
        \vspace*{3cm}
        {\fontsize{18pt}{0}
          \makebox[80pt]{实验名称:} \makebox[100pt]{\Large 图像直方图变换}\\
          \vspace*{2cm}
          \makebox[80pt]{姓\qquad \quad 名:} \makebox[100pt]{\Large xxx}\\
          \vspace*{0.5cm}
          \makebox[80pt]{学\qquad \quad 号:} \makebox[100pt]{\Large xxxx}\\
          \vspace*{0.5cm}
          \makebox[80pt]{专\qquad \quad 业:} \makebox[100pt]{\Large xxxx}\\
          \vspace*{0.5cm}
          \makebox[80pt]{实验时间:} \makebox[100pt]{\Large yyyy.mm.dd}\\
          \vspace*{5cm}
          杭州电子科技大学\\通信工程学院
        }
      \end{center}
  \end{titlepage}

  \CTEXsetup[format={\Large\bfseries}]{section}

  \newpage
  \section{实验目的}
  \begin{enumerate}
    \item 掌握灰度直方图的概念及其计算方法
    \item 掌握直方图均衡化计算过程
    \item 掌握彩色直方图的概念和计算方法
    \item 编程实现图像增强算法
  \end{enumerate}

  \section{实验内容}
  \begin{enumerate}
    \item 读取任一彩色图,根据直方图的定义编写程序按步骤实现直方图绘制与均衡化代码,
    并使用Matlab图像工具箱中的直方图计算与均衡化函数对实验结果进行验证
    \item 附加题1:调用1中编写的函数,实现彩色图像的直方图均衡化
  \end{enumerate}

  \section{算法设计}
  \paragraph*{1.简述直方图均衡化算法实现步骤\\}
  先将图像数据类型转换为uint8型,统计图像中灰度值相同的像素点的个数,
  然后将每个灰度值的个数除以像素点总数,得到每个灰度值的分布概率,
  最后对这些灰度值的分布概率做类似香农编码的处理,从0开始逐个累加,
  并且每累加一次都乘以255得到该灰度值对应的直方图均衡化后的值,即可得到一个灰度值映射曲线,
  然后将其应用到原图像就得到了直方图均衡化后的结果
  \paragraph*{2.图像直方图均衡化后有何变化,为什么?\\}
  对图像做直方图均衡化会增强图像的对比度\\
  因为直方图均衡化倾向于使图像的每个灰度值的分布概率相同,
  因此能将图像中灰度值较集中的区域分布到其他灰度等级上

  \section{运行结果与分析}

  \subsection*{显示图像直方图}
  \begin{figure}[H]
    \includegraphics*[width=1.0\textwidth]{fig/hist.png}
    \caption{左图为原图像;中图为自己实现的直方图函数;右图为调用matplotlib的库函数}
  \end{figure}

  \subsection*{直方图均衡化}
  \begin{figure}[H]
    \includegraphics*[width=1.0\textwidth]{fig/equagray.png}
    \caption{左图为原图像;中图为自己实现的直方图均衡化;右图为调用opencv的库函数}
  \end{figure}

  \subsection*{直方图均衡化变换曲线}
  \begin{figure}[H]
    \centering \includegraphics*[width=0.4\textwidth]{fig/equacurve.png}
    \caption{从曲线中可以看出,直方图均衡化对该图像起到了使图像变暗的效果}
  \end{figure}

  \subsection*{彩色图像直方图均衡化}
  \begin{figure}[H]
    \centering \includegraphics*[width=1.0\textwidth]{fig/equacolor.png}
    \caption{左图为原图像;中图为直方图均衡化后的图像;右图为变换曲线}
  \end{figure}
  从曲线中可以看出,直方图均衡化对该图像起到了使暗处变暗、亮处变亮的效果

  \section{实验小结}
  本次实验通过手动绘制直方图和实现直方图均衡化算法,
  使我对直方图均衡化算法有了更深刻的理解,并且与之前所学的信息论的知识有一定的联系,
  让我意识到了这个算法背后所蕴含的基本原理(使灰度值分布概率相同,即降低图像的信息熵),
  并且有了改进算法的思路,如使用哈夫曼编码代替香农编码

\end{document}
