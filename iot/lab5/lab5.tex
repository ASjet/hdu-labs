\documentclass[a4paper]{ctexart}
\usepackage[utf8]{inputenc}
\usepackage[a4paper]{geometry}
\usepackage{graphicx}
\usepackage{float}
\usepackage{hyperref}
\usepackage[heading = false]{ctex}
\usepackage{xcolor}
\usepackage{fontspec}
\usepackage{listings}
\usepackage{color}

\definecolor{dkgreen}{rgb}{0,0.6,0}
\definecolor{gray}{rgb}{0.5,0.5,0.5}
\definecolor{mauve}{rgb}{0.58,0,0.82}

\pagestyle{plain}
\geometry{top=1.0cm, bottom=2.0cm}
\lstset{
  frame=tb,
  aboveskip=3mm,
  belowskip=3mm,
  showstringspaces=false,
  columns=flexible,
  numbers=none,
  basicstyle = \ttfamily,
  numberstyle=\tiny\color{gray},
  keywordstyle=\color{blue},
  commentstyle=\color{dkgreen},
  stringstyle=\color{mauve},
  breaklines=true,
  breakatwhitespace=true,
  tabsize=3
}

\begin{document}
  \begin{titlepage}
      \songti
      \begin{center}
        \vspace*{1cm}
        \includegraphics[width=0.9\textwidth]{../HDU.png}\\
        \vspace*{2.5cm}
        {\fontsize{24pt}{0}
          《物联网技术实验》\\
          \fontsize{36pt}{0}
          \vspace*{1cm}
          实验报告\\
        }
        \vspace*{3cm}
        {\fontsize{18pt}{0}
          \makebox[80pt]{实验名称:} \underline{\makebox[200pt]{\Large ADC、PWM和$\mathrm{I^2C}$的应用}}\\
          \vspace*{2cm}
          \makebox[80pt]{姓\qquad \quad 名:} \underline{\makebox[100pt]{\Large xxx}}\\
          \vspace*{0.5cm}
          \makebox[80pt]{学\qquad \quad 号:} \underline{\makebox[100pt]{\Large xxxx}}\\
          \vspace*{0.5cm}
          \makebox[80pt]{专\qquad \quad 业:} \underline{\makebox[100pt]{\Large xxxx}}\\
          \vspace*{0.5cm}
          \makebox[80pt]{实验时间:} \underline{\makebox[100pt]{\Large yyyy.mm.dd}}\\
          \vspace*{5cm}
          杭州电子科技大学\\通信工程学院
        }
      \end{center}
  \end{titlepage}

  \CTEXsetup[format={\Large\bfseries}]{section}

  \newpage
  \section{实验目的}
    掌握CC3200 launchpad基本模块ADC、PWM、$\mathrm{I^2C}$的原理和应用以及Energia相关命令的使用方法。

  \section{实验内容}
    \begin{enumerate}
      \item LMT84温度采集
      \item HDC1080温湿度采集
      \item ADC原理与应用
      \item PWM的原理与应用
      \item $\mathrm{I^2C}$原理与应用
    \end{enumerate}

  \section{算法设计}
    \paragraph{1.描述ADC的四个转换过程?ADC的分类及特点?ADC的典型参数分辨率、转换速率、采样速率是什么?}
    \paragraph{ADC的四个转换过程}
    \begin{enumerate}
      \item \textbf{采样}\quad 用每隔一定时间的信号样值序列来代替原来在时间上连续的信号,
      也就是在时间上将模拟信号离散化
      \item \textbf{保持}\quad 将采样值在下次采样之前保持一段时间,以便后续处理
      \item \textbf{量化}\quad 用有限个幅度值近似原来连续变化的幅度值,
      把模拟信号的连续幅度变为有限数量的有一定间隔的离散值
      \item \textbf{编码}\quad 编码则是按照一定的规律,把量化后的值用二进制数字表示,
      然后转换成二值或多值的数字信号流
    \end{enumerate}

    \paragraph{ADC的分类}
    \begin{itemize}
      \item \textbf{积分型}\quad 优点是用简单电路就能获得高分辨率,抗干扰能力强;
      缺点是由于转换精度依赖于积分时间,因此转换速率极低。
      \item \textbf{逐次比较型}\quad 优点是速度较高、功耗低,在低分辩率(<12位)时价格便宜;
      缺点是高精度(>12位)时价格很高
      \item \textbf{并行比较型}\quad 优点是转换速率极高;
      缺点是电路规模极大,价格高,适用范围小
      \item \textbf{$\Sigma-\Delta$调制型}\quad 电路的数字部分容易单片化,容易做到高分辨率
      \item \textbf{电容阵列逐次比较型}\quad 相比普通逐次比较型可以低成本制造出高精度AD器件
      \item \textbf{压频变换型}\quad 优点是分辩率高、功耗低、价格低;
      缺点是需要外部计数电路共同完成AD转换。
    \end{itemize}

    \paragraph{ADC的典型参数}
    \begin{itemize}
      \item 分辨率:8位
      \item 转换速率:10-50$\mathrm{\mu s}$
      \item 采样速率:由奈奎斯特采样定律根据应用场景确定,一般为最高频率的2.56倍
    \end{itemize}

    \paragraph{2.温度采集原理与实现?有哪些温度传感器各有什么特点?\\}
    温度采集的原理是通过温度传感器(如热敏电阻)将温度转换为电信号,然后通过AD转换来将电信号量化为数字值,
    最后根据映射关系将量化值映射到对应的温度值。

    温度传感器可分为正系数和副系数两种,表示其读数随着温度的升高而上升或下降;也可分为线性系数和非线性系数,
    表示其读数与温度变化呈线性或非线性关系。

    \paragraph{3.描述PWM的相关参数及其之间的关系?如果我们想要控制一个直流电机的转速,
    详细描述通过PWM驱动器控制与采用可调直流电源控制相同之处和区别。\\}
    \quad PWM包括三个参数:幅度、调制频率、占空比。幅度是指PWM处于高电平时的电平值;
    占空比是一个周期内高电平占整个周期的比例;调制频率决定了PWM占空比的分辨率;
    PWM的占空比和幅度决定了其输出的功率大小。

    控制直流电机的转速本质是控制电机的输入功率。输入功率
    $$
      \mathrm{P}=\mathrm{UI}
      =\lim_{T \to 0}  \frac{1}{T}\sum_{i}^{N}P_it_i
    $$

    可调直流电源和PWM都是通过改变电机的功率来控制转速。
    但是直流电源是通过控制输出的电压值和电流值来改变功率;PWM是通过控制占空比来改变周期内的平均功率。

    \paragraph{4.HDC1080有哪些特性?包含哪些硬件资源?应用范围?}
    \begin{itemize}
      \item 相对湿度精度为 ±2\%(典型值)
      \item 温度精度为 ±0.2°C(典型值)
      \item 高湿度下具有出色的稳定性
      \item 14 位测量分辨率
      \item 睡眠模式的电流为 100nA
      \item 平均电源电流:
      \begin{itemize}
        \item 1sps、11 位相对湿度 (RH) 测量时为 710nA
        \item 1sps、11 位 RH 与温度测量时为 1.3µA
      \end{itemize}
      \item 电源电压范围:2.7V 至 5.5V
      \item 3mm x 3mm 小型器件封装
      \item $\mathrm{I^2C}$ 接口
    \end{itemize}
    \paragraph*{应用范围}
    \begin{itemize}
      \item 制热、通风与空调控制 (HVAC)
      \item 智能温度调节装置和室温监视器
      \item 大型家用电器
      \item 打印机
      \item 手持式计量表
      \item 医疗设备
      \item 无线传感器(TIDA:00374、00484、00524)
    \end{itemize}

    \paragraph{5.$\mathrm{I^2C}$工作原理?有什么特点?速率模式?描述$\mathrm{I^2C}$通讯协议以及读写时序。\\}
    \quad $\mathrm{I^2C}$全称为Inter-Integrated Circuit bus,即集成电路总线。
    $\mathrm{I^2C}$是一种串行通信总线,使用多主从架构,
    由飞利浦公司在1980年代为了让主板、嵌入式系统或手机用以连接低速周边设备而发展。
    \paragraph*{速率模式}
    \begin{itemize}
      \item 低速模式(10 kbit/s)
      \item 标准模式(100 kbit/s)
      \item 快速模式(400 kbit/s)
      \item 快速+模式(1 Mbit/s)
      \item 高速模式(3.4 Mbit/s)
      \item 超高速模式(5 Mbit/s)
    \end{itemize}

    \paragraph*{常用读写时序}
    \begin{itemize}
      \item (启始)-[控制]-[指令]-[资料]-(结束)
      \item (启始)-[控制0]-[指令]-(r启始)-[控制1]-[资料]-(结束)
    \end{itemize}

    \paragraph{6.设计采用HDC1080采集温湿度数据,并通过$\mathrm{I^2C}$传输数据。当温湿度超过阈值(T:28°C/H:40\%),
    通过UART发出警报,并点亮对应的警报灯。给出程序设计流程图和详细程序、运行结果及结果分析。\\}
    \includegraphics*[width=1.0\textwidth]{iot-lab5.pdf}
    \lstinputlisting[language=C++]{temp_humi/temp_humi.ino}

    \begin{figure}[H]
      \begin{tabular}{p{0.5\textwidth}p{0.5\textwidth}}
        \centering
        \includegraphics*[width=0.5\textwidth]{fig/uart_underthres.png}
        \caption{温湿度未超过阈值,UART输出正常}
        &
        \centering
        \includegraphics*[width=0.5\textwidth]{fig/uart_overthres.png}
        \caption{温湿度超过阈值,UART输出警报}
      \end{tabular}

      \begin{tabular}{p{0.5\textwidth}p{0.5\textwidth}}
        \centering
        \includegraphics*[width=0.5\textwidth]{fig/underthres.jpg}
        \caption{温度未超过阈值,红灯不亮}
        &
        \centering
        \includegraphics*[width=0.5\textwidth]{fig/overthres.jpg}
        \caption{温度超过阈值,红灯亮起}
      \end{tabular}

    \end{figure}


  \section{实验小结}
  这次实验使用了CC3200的I2C总线和AD转换模块,让我通过温度采集的过程了解了在嵌入式设备中进行模数转换的整个流程。

\end{document}
