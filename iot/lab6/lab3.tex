\documentclass[a4paper]{ctexart}
\usepackage[utf8]{inputenc}
\usepackage[a4paper]{geometry}
\usepackage{graphicx}
\usepackage{float}
\usepackage{hyperref}
\usepackage[heading = false]{ctex}
\usepackage{xcolor}
\usepackage{fontspec}
\usepackage{listings}
\usepackage{color}

\definecolor{dkgreen}{rgb}{0,0.6,0}
\definecolor{gray}{rgb}{0.5,0.5,0.5}
\definecolor{mauve}{rgb}{0.58,0,0.82}

\pagestyle{plain}
\geometry{top=1.0cm, bottom=2.0cm}
\lstset{
  frame=tb,
  aboveskip=3mm,
  belowskip=3mm,
  showstringspaces=false,
  columns=flexible,
  basicstyle = \ttfamily,
  numbers=none,
  numberstyle=\tiny\color{gray},
  keywordstyle=\color{blue},
  commentstyle=\color{dkgreen},
  stringstyle=\color{mauve},
  breaklines=true,
  breakatwhitespace=true,
  tabsize=3
}

\begin{document}
  \begin{titlepage}
      \songti
      \begin{center}
        \vspace*{1cm}
        \includegraphics[width=0.9\textwidth]{../HDU.png}\\
        \vspace*{2.5cm}
        {\fontsize{24pt}{0}
          《物联网技术实验》\\
          \fontsize{36pt}{0}
          \vspace*{1cm}
          实验报告\\
        }
        \vspace*{3cm}
        {\fontsize{18pt}{0}
          \makebox[80pt]{实验名称:} \underline{\makebox[230pt]{\Large SPI和WIFI的应用}}\\
          \vspace*{2cm}
          \makebox[80pt]{姓\qquad \quad 名:} \underline{\makebox[100pt]{\Large xxx}}\\
          \vspace*{0.5cm}
          \makebox[80pt]{学\qquad \quad 号:} \underline{\makebox[100pt]{\Large xxxx}}\\
          \vspace*{0.5cm}
          \makebox[80pt]{专\qquad \quad 业:} \underline{\makebox[100pt]{\Large xxxx}}\\
          \vspace*{0.5cm}
          \makebox[80pt]{实验时间:} \underline{\makebox[100pt]{\Large yyyy.mm.dd}}\\
          \vspace*{5cm}
          杭州电子科技大学\\通信工程学院
        }
      \end{center}
  \end{titlepage}

  \CTEXsetup[format={\Large\bfseries}]{section}

  \newpage
  \section{实验目的}
    掌握CC3200 launchpad基本模块SPI、WIFI的原理和应用以及Energia相关命令的使用方法

  \section{实验内容}
    \begin{enumerate}
      \item LDC1000 电感采集
      \item HDC1080 温湿度网络采集
      \item SPI原理与应用
      \item WIFI的原理与应用
    \end{enumerate}

  \section{运行结果与分析}
    \subsection*{1.描述SPI的读写过程?SPI的工作机制与特点?}
      SPI设备之间通信需要四根数据线:MISO(Master In Slave Out)、MOSI(Master Out Slave In)、
      SCK(Serial ClockD)、SS(Slave Select)
      \begin{itemize}
        \item MISO:从设备数据发送线,主设备使用该数据线实现数据读取
        \item MOSI:主设备数据发送线,从设备通过该数据线获取数据
        \item SCK:时钟信号线,同步主从设备之间的数据发送和读取
        \item SS:片选线,低电平有效,该数据线处于低电平时。主从设备之间传输数据有效,即主
        设备发出的数据从设备能够读取成功
      \end{itemize}
      SPI 在数据通信过程基本时序为:1、SS(CS)置低电平,使数据通信有效;2、进行数据读取和发送;
      3、SS(CS)置高电平,结束本次数据通信过程。
      \paragraph*{SPI的特点}
      \begin{itemize}
        \item 可以同时发出和接收串行数据
        \item 可以当作主机或从机工作
        \item 提供频率可编程时钟
        \item 发送结束中断标志
        \item 写冲突保护
        \item 总线竞争保护等
      \end{itemize}
    \subsection*{2.WIFI应用过程中,主要需要涉及哪些参数?请结合所需要的函数简要说明}
    \paragraph*{主要涉及的参数}
    \begin{itemize}
      \item SSID,表示WiFi网络的名称
      \item password,表示访问WiFi网络所需的认证口令
      \item 加密方式,即WiFi所使用的加密方式,包括WEP、WPA/WPA2等
    \end{itemize}
    \newpage
    \subsection*{3.通过网络读取HDC1080和LDC1000的值,CC3200设置为webserver模式,用户可以通过浏览器
    访问CC3200,实验过程中CC3200的IP地址可以通过串口打印至PC端。请写出程序流程图,给出程序和结果}

    \begin{figure}[H]
      \includegraphics*[width=1.0\textwidth]{lab3flowchart.pdf}
      \caption{程序流程图}
    \end{figure}

    \paragraph*{程序代码\\}
    \lstinputlisting[language=C++]{webserver/webserver.ino}

    \begin{figure}[H]
      \includegraphics*[width=1.0\textwidth]{res_uart.png}
      \caption{串口输出}
      \includegraphics*[width=0.5\textwidth]{res_http.png}
      \caption{webserver响应}
    \end{figure}

  \section{实验小结}
  本次实验使用了SPI和WIFI模块,让我了解了SPI的原理和用法,以及使用CC3200连接WiFi的方法。
  在实验过程中实现了一个简单的HTTP服务器,用于通过WiFi网络远程读取CC3200的模块读数,
  让我对物联网的结构体系有了一个大体上的认识。

\end{document}
