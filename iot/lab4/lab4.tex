\documentclass[a4paper]{ctexart}
\usepackage[utf8]{inputenc}
\usepackage[a4paper]{geometry}
\usepackage{graphicx}
\usepackage{float}
\usepackage{hyperref}
\usepackage[heading = false]{ctex}
\usepackage{xcolor}
\usepackage{fontspec}
\usepackage{listings}
\usepackage{color}

\definecolor{dkgreen}{rgb}{0,0.6,0}
\definecolor{gray}{rgb}{0.5,0.5,0.5}
\definecolor{mauve}{rgb}{0.58,0,0.82}

\pagestyle{plain}
\geometry{top=1.0cm, bottom=2.0cm}
\lstset{
  frame=tb,
  aboveskip=3mm,
  belowskip=3mm,
  showstringspaces=false,
  columns=flexible,
  numbers=none,
  basicstyle = \ttfamily,
  numberstyle=\tiny\color{gray},
  keywordstyle=\color{blue},
  commentstyle=\color{dkgreen},
  stringstyle=\color{mauve},
  breaklines=true,
  breakatwhitespace=true,
  tabsize=3
}

\begin{document}
  \begin{titlepage}
      \songti
      \begin{center}
        \vspace*{1cm}
        \includegraphics[width=0.9\textwidth]{../HDU.png}\\
        \vspace*{2.5cm}
        {\fontsize{24pt}{0}
          《物联网技术实验》\\
          \fontsize{36pt}{0}
          \vspace*{1cm}
          实验报告\\
        }
        \vspace*{3cm}
        {\fontsize{18pt}{0}
          \makebox[80pt]{实验名称:} \underline{\makebox[230pt]{\Large GPIO、外部中断和UART的应用}}\\
          \vspace*{2cm}
          \makebox[80pt]{姓\qquad \quad 名:} \underline{\makebox[100pt]{\Large xxx}}\\
          \vspace*{0.5cm}
          \makebox[80pt]{学\qquad \quad 号:} \underline{\makebox[100pt]{\Large xxxx}}\\
          \vspace*{0.5cm}
          \makebox[80pt]{专\qquad \quad 业:} \underline{\makebox[100pt]{\Large xxxx}}\\
          \vspace*{0.5cm}
          \makebox[80pt]{实验时间:} \underline{\makebox[100pt]{\Large yyyy.mm.dd}}\\
          \vspace*{5cm}
          杭州电子科技大学\\通信工程学院
        }
      \end{center}
  \end{titlepage}

  \CTEXsetup[format={\Large\bfseries}]{section}

  \newpage
  \section{实验目的}
  掌握CC3200 launchpad基本模块GPIO、外部中断、UART的原理和应用以及Energia相关命令的使用方法。

  \section{实验内容}
  \begin{enumerate}
    \item Energia基础及安装
    \item CC3200 launchpad相关硬件
    \item GPIO原理与应用
    \item 外部中断的原理与应用
    \item UART原理与应用
  \end{enumerate}

  \section{运行结果与分析}
  \paragraph{1.描述什么是GPIO中断?中断源有哪些?中断分类?中断系统一般包括那些部分?如何设置中断优先级?\\}

  \quad GPIO全程为General Purpose Input/Output,即通用输入输出接口,能通过程序读取和控制引脚的电平值。
  GPIO中断就是由GPIO引脚的输入引发的中断。
  CC3200 具有诸多中断源:GPIO 中断,Timer 中断,SPI 中断,I2C 中断等。

  中断可以分为外部中断和内部中断。中断系统一般由硬件部分的状态控制字(控制寄存器)和软件部分的中断向量组成。

  在CC3200中可以通过CC3200SDK编程设置中断优先级

  \paragraph{2.为什么需要对按键进行消抖?按键消抖有哪些方法?各有什么优缺点?\\}

  \quad 因为按键的非刚体机械结构会导致触点在接触的瞬间会多次通断,导致电平上下延有毛刺出现,
  即按键抖动。由于处理器的指令周期在微秒级,而按键抖动的持续时间在毫秒级,
  因此为了保证程序能正确的识别出按键事件而不被抖动干扰,需要对按键进行消抖。

  \paragraph{按键消抖包括硬件消抖和软件消抖}
  \begin{itemize}
    \item 硬件消抖是使用消抖电路(双稳态寄存器、RC滤波等)实现,
    使得处理器收到的信号是消抖过的信号。优点是不需要再在软件层面上进行处理,性能好;
    缺点是当按键数过多时会导致电路复杂度上升。
    \item 软件消抖是使用延时的方法,在第一次收到按键信号时延时等待一段时间(5-20ms)之后再次读取按键电平值,
    判断是否发生电平改变,以确定按键事件。优点是可以处理所有的按键抖动;缺点是延时等待会浪费CPU周期。
  \end{itemize}

  \paragraph{3.MCU是什么?MCU分为哪几类?各有什么特点和应用场合?\\}

  \quad MCU全称是Micro Controller Unit,即微控制器单元,
  是指将处理器和各种外围设备(内存、UART、A/D等)集成到一块芯片上,形成的芯片级计算机,即单片机。

  \begin{itemize}
    \item 按用途可以分为通用型MCU和专用型MCU
    \item 按处理器位宽可分为4位、8位、16位、32位和64位MCU
    \item 按存储器结构可分为哈佛型和冯诺依曼型
    \item 按指令集架构可分为精简指令集(RISC)型和复杂指令集(CISC)型
  \end{itemize}

  \paragraph{4.CC3200有哪些特性?包含哪些硬件资源?应用范围?\\}
  CC3200 SimpleLink Wi-Fi — 由应用微控制器,Wi-Fi 网络处理器和电源管理子系统组成
  \paragraph*{硬件资源如下}
  \begin{itemize}
    \item 应用微控制器子系统
    \begin{itemize}
      \item ARM®Cortex®-M4 内核,运行频率 80MHz
      \item 嵌入式存储器
      \begin{itemize}
        \item RAM(高达 256KB)
        \item 外部串行闪存引导加载程序,和 ROM 中的外设驱动程序
      \end{itemize}
      \item 32 通道直接内存访问 (DMA)
      \item 针对高级快速安全性的硬件加密引擎,其中包括
      \begin{itemize}
        \item AES,DES 和 3DES
        \item SHA2 和 MD5
        \item 循环冗余校验 (CRC) 与校验和
      \end{itemize}
      \item 8位并行摄像头接口
      \item 1个多通道音频串口 (McASP) 接口,支持 2 个 I2S 通道
      \item 1 个 SD/MMC 接口
      \item 2 个通用异步收发器 (UART)
      \item 1 个串行外设接口 (SPI)
      \item 1 个内部集成电路 ($\mathrm{I^2C}$)
      \item 4 个通用定时器,支持 16 位脉宽调制 (PWM) 模式
      \item 4 个通用定时器,支持 16 位脉宽调制 (PWM) 模式
      \item 4 通道 12 位模数转换器 (ADC)
      \item 多达 27 个独立可编程、复用通用输入输出 (GPIO) 引脚
    \end{itemize}
    \item Wi-Fi 网络处理器子系统
    \begin{itemize}
      \item 特有 Wi-Fi Internet-On-a-Chip™
      \item 专用 ARM MCU 完全解除应用微控制器的 Wi-Fi 和互联网协议处理负担
      \item ROM 中 的 Wi-Fi 以及 互联网协议
      \item 802.11 b/g/n 射频、基带,媒介访问控制 (MAC),Wi-Fi 驱动器和请求方
      \item TCP/IP 堆栈
      \begin{itemize}
        \item 行业标准 BSD 插槽应用编程接口 (API)
        \item 8 个同时 TCP 或 UCP 插槽
        \item 2 个同时 TLS 和 SSL 插槽
      \end{itemize}
      \item 强大的加密引擎,用于与针对 TLS 和 SSL 连接的 256 位 AES 加密的快速、安全 Wi-Fi 和互联网连接
      \item 基站、访问点 (AP) 和 Wi-Fi Direct® 模式
      \item WPA2 个人和企业安全性
      \item 针对自主和快速 Wi-Fi 连接的 SimpleLink 连接管理器
      \item SmartConfig™ 技术,AP 模式和 WPS2,这些技术用于实现简单且灵活的 Wi-Fi 服务开通
      \item Tx 功率
      \begin{itemize}
        \item 18.0 dBm @ 1 DSSS
        \item 18.0 dBm @ 1 DSSS
      \end{itemize}
      \item RX 灵敏度
      \begin{itemize}
        \item -95.7 dBm @ 1 DSSS
        \item -74.0 dBm @ 54 OFDM
      \end{itemize}
    \end{itemize}
    \item 电源管理子系统
    \begin{itemize}
      \item 集成直流-直流转换器支持宽范围的电源电压:
      \begin{itemize}
        \item $\mathrm{V_{BAT}}$ 宽电压模式:2.1 至 3.6V
        \item 预稳压 1.85V 模式
      \end{itemize}
      \item 高级低功耗模式
      \begin{itemize}
        \item 支持实时时钟 (RTC) 的休眠:4µA
        \item 低功耗深度睡眠 (LPDS):120 µA
        \item RX 流量(MCU 激活):59 mA@54正交频分复用 (OFDM)
        \item TX 流量(MCU 激活):229 mA@54OFDM,最大功率
        \item 空闲连接(处于 LPDS 中的 MCU):695 µA @ DTIM = 1
      \end{itemize}
    \end{itemize}
    \item 时钟源
    \begin{itemize}
      \item 具有内部振荡器的 40.0MHz 晶振
      \item 32.768kHz 晶振或外部 RTC 时钟
    \end{itemize}
    \item 封装和工作温度
    \begin{itemize}
      \item 0.5mm 焊球间距,64 引脚,9mm x 9mm 四方扁平无引线 (QFN)
      \item 环境温度范围:-40°C 至 85°C
    \end{itemize}
  \end{itemize}

  \paragraph*{主要应用于物联网范围,比如}
  \begin{itemize}
    \item 云连通性
    \item 家庭自动化
    \item 家用电器
    \item 访问控制
    \item 安防系统
    \item 智能能源
    \item 互联网网关
    \item 工业控制
    \item 智能插座和仪表计量
    \item 无线音频
    \item IP网络传感器节点
  \end{itemize}

  \paragraph{5.UART工作原理?有什么特点?描述UART通讯协议。假设串口波特率为9600,则一个byte实际传输速率为多少?\\}
  \quad UART全称是Universal Asynchronous Receiver/Transmitter,即通用异步收发传输器,
  将数据由串行通信与并行通信间做传输转换,作为并行输入称为串行输出的芯片。

  UART是一种通用串行数据总线,用于异步通信。该总线双向通信,可以实现全双工传输和接收。
  \paragraph*{UART通信协议}
  \begin{itemize}
    \item \textbf{起始位}:先发出一个逻辑“0”的信号,表示传输字符开始。
    \item \textbf{数据位}:紧接着起始位之后。数据位的个数可以是4、5、6、7、8等,构成一个字符。
    字符通常采用ASCII码来表示。从最低位开始传送,靠时钟定位。
    \item \textbf{奇偶校验位}:数据位加上这一位后,使得“1”的位数应为偶数(偶校验)或奇数(奇校验),
    以次来校验数据传送的正确性。
    \item \textbf{停止位}:它是一个字符数据的结束标志。可以是1位、1.5位、2位的高电平。
    由于数据是在传输线上定时的,并且每一个设备有其自己的时钟,很可能在通信中两台设备间出现不同步。
    因此停止位不仅仅是表示传输的结束,并且提供计算机校正时钟同步的机会。
    适用于停止位的位数越多,不同时钟同步的容忍程度越大,但是数据传输率也就越慢。
    \item \textbf{空闲位}:处于逻辑"1"状态,表示当前线路上没有数据传输。
  \end{itemize}

  当串口波特率为9600时,传输一个Byte的速率为$8/9600\approx 0.83\mathrm{ms}$

  \paragraph{6.设计带控制开关的交通灯,红灯停、黄灯等、绿灯行,若有行人按开关,绿灯亮,行人优先通过。
  通过UART返回对应灯的四种通行状态。给出程序设计流程图和详细程序,运行结果及结果分析\\}
  \includegraphics*[width=1.0\textwidth]{iot-lab1.pdf}
  \lstinputlisting[language=C++]{code/code.ino}
  程序中设置了红灯持续时间为2秒,绿灯的持续时间为2秒,黄灯的持续时间为0.5秒。
  每当行人按下按键时,绿灯会亮起并且持续2秒,然后从红灯开始继续循环。
  若行人在绿灯亮时按下按钮,会导致绿灯的计时重置。

  \includegraphics*[width=1.0\textwidth]{uart.png}

  \section{实验小结}
  这次实验使用UART和按键中断完成了交通灯控制程序,让我对中断的作用有了更深的理解。

\end{document}
